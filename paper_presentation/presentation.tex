\documentclass[10pt]{beamer}
\usetheme[progressbar=frametitle]{metropolis}
\usepackage[utf8]{inputenc}
\usepackage[T1]{fontenc}
\usepackage[french,english]{babel}
\usepackage{booktabs}
\usepackage[scale=2]{ccicons}
\usepackage{picture}
\usepackage{pgfplots}
\usepgfplotslibrary{dateplot}
\usepackage{bm}
\usepackage{mathtools}
\usepackage{subcaption}
\usepackage{comment}
\usepackage{xspace}
\newcommand{\themename}{\textbf{\textsc{metropolis}}\xspace}
\usepackage{adjustbox}
\makeatletter
\setlength{\metropolis@titleseparator@linewidth}{1pt}
\setlength{\metropolis@progressonsectionpage@linewidth}{1pt}
\setlength{\metropolis@progressinheadfoot@linewidth}{1pt}
\makeatother
\definecolor{Purple}{HTML}{911146}
\definecolor{bluecerdi}{HTML}{32567c} 
\definecolor{Green}{RGB}{168,187,33}
\definecolor{redcerdi}{HTML}{7c324f} 
\definecolor{bluered}{RGB}{18, 73, 110}
\definecolor{UBCblue}{rgb}{0.04706, 0.13725, 0.26667} % UBC Blue (primary)
\definecolor{UBCgrey}{RGB}{140, 0, 0} 
\definecolor{orange2}{RGB}{174, 54, 5}
\usefonttheme{default}   
\setbeamercolor{background canvas}{bg=white}
\setbeamercolor{alerted text}{fg=bluecerdi}
\setbeamercolor{frametitle}{fg=UBCblue}
\setbeamercolor{frametitle}{bg=white}
\setbeamertemplate{itemize items}[default]  
\setbeamerfont{frametitle}{size=\normalsize}
\usepackage{ragged2e}  
\justifying 
%\let\olditem\item
%\renewcommand\item{\olditem\justifying}
\usepackage{appendixnumberbeamer}
\setbeamercolor{button}{bg=lightgray,fg=black}
\usetikzlibrary{shapes.geometric, arrows.meta, positioning}
\renewcommand{\raggedright}{\leftskip=0pt \rightskip=0pt plus 0cm}


\usepackage{times}

\title{\textbf{SDD - Special Issue : Reproducibility and Project Organisation in Research}\\[0.2em]\Large{\textit{}}}
%\subtitle{None}
% \date{\today}
\date{\today}
\author{The R Lobby (Flo \& Alex)}
\institute{\bigskip \bigskip \includegraphics[scale=0.25]{paper_presentation/figures/Dauphine_LOGO.png}\centering}
%\titlegraphic{\centering  \includegraphics[height=1.1cm]{Figures/logo.png} \includegraphics[height=0.9cm]{Figures/Logo_CES_fond-blanc}\includegraphics[height=1cm]{Figures/pse.jpg}}
\begin{document}

\begin{frame}[plain]

    \maketitle
    
\end{frame}

\begin{frame}{Vue d'ensemble}
\tableofcontents
\end{frame}

\section{Project organisation of a research project}

\begin{frame}\frametitle{Folder structure and R project}


\alert{Un nombre de réfugiés croissant en France comme dans le monde.}
\begin{itemize}
    \item {Mondial} : Niveaux records du nombre de déplacés en 2023 (HCR). 
    \item {Europe} : De 7 à 12,4 millions de réfugiés  entre 2021 et 2022.
    \item {France} : Demandes de protection internationale passées de 104 577 en 2021 à 142 500 en 2023 (OFPRA).
\end{itemize}

\alert{Une population particulièrement sujette aux troubles mentaux.}

\begin{itemize}
    \item Des expériences d’événements violents et de danger de mort dans le pays d’origine ou au cours de la fuite + des conditions d’accueil dégradées ont des effets durables sur la santé mentale des réfugiés.

    \item Ces troubles affectent la capacité des réfugiés à s'intégrer durablement.
\end{itemize}
\end{frame}

\begin{frame}{Quarto \textit{v.s.} Rmarkdown}

    
\end{frame}

\section{Github integration in R}

\begin{frame}{What is Git? What is GitHub?}

\begin{itemize}
    \item Git: an open source software for version control 
    \item GitHub: service for collaborating on code using Git 
\end{itemize}


Using Git/GitHub: 
\begin{itemize}
    \item serves as a backup 
    \item allows you to use version control 
    \item makes it possible to work on the same project at the same time as collaborators 

\end{itemize}

Source: \href{https://rfortherestofus.com/2021/02/how-to-use-git-github-with-r}{R for the rest of us' blogpost about the Github integration in R}

    
\end{frame}

\begin{frame}{How to setup Git}

See: \href{https://rfortherestofus.com/2021/02/how-to-use-git-github-with-r}{R for the rest of us' blogpost about the Github integration in R}

\begin{itemize}
    \item \href{https://happygitwithr.com/install-git.html }{Install Git}
    \item \href{https://happygitwithr.com/hello-git.html}{Configure Git}  
    \item Initialize a Git Repository
    \begin{itemize}
        \item New Project > New Directory > Create Project 
        \item in Console: library(usethis), use\_git() > creates Git tab in environment 
        \item You can view make commits and view history (for version control)
    \end{itemize}

\end{itemize}


\end{frame}

\begin{frame}{Github integration in R}

First, connect RSTudio \& GitHub : 
\begin{itemize}
    \item Create GitHub account then create Personal Access Token on GitHub (save it!)
    \item Sync RStudio \& GitHub : library(gitcreds), gitcreds\_set() -> enter PAT 
\end{itemize}


In Github -> create repository
In R : New project -> Version Control -> Git -> copy repository URL + create clone local repository 
    
\end{frame}

\section{Github integration in Overleaf}

\section{Discussion around each others' good practices}

\section{Some useful links}
\begin{frame}{Some useful links}
    
\alert{Reproducibility}
\begin{itemize}
    \item \href{https://github.com/FedericoTartarini/reproducible-research}{Federico Tartarini's great repository around reproducible research organisation}
    \item ...
\end{itemize}

\alert{Github with R}
\begin{itemize}
    \item \href{https://rfortherestofus.com/2021/02/how-to-use-git-github-with-r}{R for the rest of us' blogpost about the Github integration in R}
    \item ...
\end{itemize}
\end{frame}





\end{document}